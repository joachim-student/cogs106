\begin{frame}
\titlepage
\end{frame}

\section{Test-driven development}

\frame{
  \frametitle{What is test-driven development (TDD)?}
  TDD is a software development methodology where tests are written for a feature before writing the code for that feature.

  \pause
  
  Yes that sounds weird, but let's give the computer scientists some credit.  We actually do it all the time intuitively.

}

\frame{
  \frametitle{Steps of TDD}
  \begin{itemize}
    \item[1.] Write a failing test.
    \item[] The test fails because the code it tests doesn't exist yet or it is not implemented yet.
    \item[2.] Write just enough code to make the test pass.
    \item[] Write the minimum amount of code necessary to pass the test. Don't add any additional features or improvements.
    \item[3.] Refactor the code as needed.
    \item[] Improve design, implementation, maintainability, \emph{all the while constantly testing} to make sure that all tests still pass.
  \end{itemize}
}

\frame{
  \frametitle{Two kinds of tests}
  
  By writing tests first and then writing the code to make the tests pass, you get better code quality, improved bug detection, and a more enjoyable development experience.
  
  \pause\vspace{2ex}

  \begin{block}{Unit tests}
  Tests individual pieces of code, such as functions or methods, in isolation.
  \end{block}
  \begin{block}{Integration tests}
  Tests how multiple pieces of code work together.
  \end{block}
}




\begin{frame}
\frametitle{TDD in practice: An example}
\begin{itemize}
\item Consider a simple example of a function that performs a statistical calculation (e.g., linear regression).
\item Before writing any production code, write test cases that test the expected behavior of the function
\begin{itemize}
\item Function gives correct output with known inputs.
\item Function correctly verifies input (throwing errors for bad input is desirable behavior!)
\end{itemize}
\item Write the function, making sure it passes all the test cases.
\item Refactor the code as needed, making sure all tests still pass.
\item Repeat the process for any additional features or bug fixes.
\end{itemize}
\end{frame}

\begin{frame}
\frametitle{Benefits of TDD}
\begin{itemize}
\item TDD helps to ensure code correctness and reduce the number of bugs.
\item It makes it easier to make changes to code and add new features, as any breaking changes will be caught by the tests.
\item TDD improves code maintainability by providing a clear and complete set of tests for the code.
\item It also helps to improve the development process by making it more incremental and iterative, allowing for a faster development cycle.
\end{itemize}
\end{frame}


\section{Cognitive modeling is TTD}

\begin{frame}
\frametitle{TDD in cognitive model development}
\begin{itemize}
\item Cognitive modeling is amateur software development.
\item Like any software development, it benefits from TDD.
\item Tests verify correctness of a model's predictions and behavior.
\item Once TDD is implemented, we can go through rapid cycles of model development:
\begin{itemize}
\item Make it easier to make changes or add new features (to the \emph{cognitive model}, not just software).
\item Ensure the reliability and robustness of a model.
\item Continuous testing cycles facilitate rapid model development.
\end{itemize}
\end{itemize}
\end{frame}

\begin{frame}
\frametitle{TDD in cognitive model development}
Examples of tests for a cognitive model could include: 
     \begin{itemize}
     \item Comparison of model predictions against observed data: does the model (still) fit?  Does it now fit better, or worse?
     \item Comparison of model predictions against common sense.
     \item Testing the model's robustness when we change...
     \begin{itemize}
     \item independent variables
     \item dependent variables
     \item convenience assumptions we may have made
     \item structural relationships within the model
     \item add entirely new features
     \item ...
     \end{itemize}
     \item Often the test can be evaluated informally (e.g., looking at a figure of model vs.\ data), but ideally it is automated.
  \end{itemize}

\end{frame}


\section{TDD workflow}

\begin{frame}
\frametitle{TDD workflow start}
\begin{enumerate}
\item \textbf{Write a test:} Write a test that defines a desired behavior or feature of the code you are developing.
\item \textbf{Watch it fail:} Run the test and see that it fails, as the desired behavior or feature has not been implemented yet.
\item \textbf{Write the code:} Write the minimum amount of code needed to make the test pass.
\item \textbf{Watch it pass:} Run the test and see that it passes, as the desired behavior or feature has now been implemented.
\item \textbf{Refactor:} Refactor the code as needed, making sure all tests continue to pass.
\end{enumerate}
\end{frame}

\begin{frame}
\frametitle{TDD workflow continue}
\begin{enumerate}
\item[6.] \textbf{Repeat the process:} Repeat the process of adding a test, running all tests, writing the code, running all tests, and refactoring as needed for each desired behavior or feature.
\item[7.] \textbf{Continuously verify:} Continuously verify the correctness of the code through the tests.
\item[8.] \textbf{Maintaining a test suite:} Maintaining a comprehensive test suite that covers all desired behaviors and features of the code.
\end{enumerate}
\end{frame}

