\documentclass[handout]{beamer}
\usetheme{metropolis}
\usepackage{colortbl}

\usetheme{metropolis}
\usepackage{listings}

\lstdefinestyle{Python}
{
    language=Python,
    basicstyle=\ttfamily\scriptsize,
    keywordstyle=\color{blue}\ttfamily,
	otherkeywords={self,False},
    commentstyle=\color{green!60!black}\ttfamily,
    stringstyle=\color{red!80!black}\ttfamily,
    backgroundcolor=\color{blue!10!white},
    showstringspaces=false,
    numbers=none,
    numberstyle=\tiny,
    numbersep=5pt,
    xleftmargin=2pt,
    framexleftmargin=4pt,
    framexrightmargin=4pt,
    framexbottommargin=5pt,
    framextopmargin=5pt,
    %breaklines=true,
    captionpos=b
}

\lstdefinestyle{MATLAB}{
    language=MATLAB,
    basicstyle=\ttfamily\scriptsize,
    keywordstyle=\color{blue}\ttfamily,
    stringstyle=\color{red!80!black}\ttfamily,
    commentstyle=\color{green!60!black}\ttfamily,
    numbers=none,
    backgroundcolor=\color{orange!15!white},
    showstringspaces=false,
    numbers=none,
    numberstyle=\tiny,
    numbersep=5pt,
    xleftmargin=2pt,
    framexleftmargin=4pt,
    framexrightmargin=4pt,
    framexbottommargin=5pt,
    framextopmargin=5pt,
    %breaklines=true,
    captionpos=b
}

\renewcommand{\emph}[1]{{\color{red}#1}}

\beamerdefaultoverlayspecification{<+->}

\setbeamercolor{block body}{bg=mDarkTeal!30}
\setbeamercolor{block title}{bg=mDarkTeal,fg=black!2}

\title{Test-Driven Development}
\author{Joachim Vandekerckhove}
\date{}


\title{Intermediate object-oriented programming}
\author{Joachim Vandekerckhove}
\date{}

\begin{document}

\maketitle


\section{Input validation}

\begin{frame}[fragile]{Input Validation}
    \begin{itemize}
        \item Input validation is the process of checking if input data is valid before using it.
        \item It helps ensure that your program runs correctly and securely.
        \item Common types of input validation include range checking, type checking, format checking, and existence checking.
    \end{itemize}
\end{frame}

\begin{frame}[fragile]{Benefits of Input Validation}
    \begin{itemize}
        \item Helps prevent unexpected behavior.
        \item Improves code readability and maintainability.
        \item Reduces security risks.
        \item Provides better user experience by giving informative error messages.
    \end{itemize}
\end{frame}

\begin{frame}[fragile]{Common Types of Input Validation}
    \begin{itemize}
        \item Range checking: Ensures input falls within acceptable range of values.
        \item Type checking: Ensures input is of the expected type.
        \item Format checking: Ensures input conforms to a specific format (e.g., phone numbers or email addresses).
        \item Existence checking: Ensures input is not empty or null.
    \end{itemize}
\end{frame}


\begin{frame}[fragile]{Basic Input Validation Example}
\begin{lstlisting}[style=matlab]
classdef BankAccount
    properties
        balance
    end
    methods
        function obj = BankAccount(balance)
            if balance < 0
                error("Balance cannot be negative.")
            end
            obj.balance = balance;
        end
    end
end
\end{lstlisting}
\end{frame}


\begin{frame}[fragile]{Handling Invalid Inputs Gracefully}
    \begin{itemize}
        \item Provide informative error messages to the user.
        \item Log error messages for developers to review.
        \item Raise exceptions with descriptive error messages.
        \item Ensure that the program does not continue with invalid input.
    \end{itemize}
\end{frame}

\begin{frame}[fragile]{Input Validation in the Context of OOP}
    \begin{itemize}
        \item Input validation can be incorporated into methods and class constructors.
        \item It's a good practice to perform input validation as close to the source of input as possible.
        \item Encapsulation allows you to define rules for valid state and prevent invalid input from corrupting the state of the object.
    \end{itemize}
\end{frame}

\section{Corruption}

\begin{frame}[fragile]{Corrupted State}
    
    A corrupted state is a situation where an object's data is in an inconsistent or invalid state, often caused by an error in the program or a violation of the object's invariants.
    
    \begin{itemize}
        \item An object is in a corrupted state when its internal data is inconsistent with the object's intended behavior
        \item A corrupted state can lead to bugs and errors in the program
        \item Preventing corrupted states is an important part of writing correct, reliable code
    \end{itemize}
    
\end{frame}


\begin{frame}[fragile]{Example: Workflow that leads to a Corrupted State}

Consider a bank account object, with a balance attribute and a deposit and withdraw method.

\begin{lstlisting}[style=MATLAB]
classdef BankAccount
  properties
    balance
  end
  methods
    function obj = BankAccount(balance)
      if balance < 0, error("Balance must be positive."), end
      obj.balance = balance;
    end
    function obj = deposit(obj, amount)
      obj.balance = obj.balance + amount;
    end
    function obj = withdraw(obj, amount)
      if amount > obj.balance, error("Overdrawing error!"), end
      obj.balance = obj.balance - amount;
    end
  end
end
\end{lstlisting}

\end{frame}

\begin{frame}[fragile]{Example: Workflow that leads to a Corrupted State}
        
    Suppose we have the following workflow:
    
    \begin{enumerate}
        \item Create a new bank account with initial balance of 100 dollars
        \item Make a deposit of -50 dollars
        \item That's arguably not a deposit
        \item This workflow violates the invariants of the bank account object and results in a corrupted state, where the balance is negative.
    \end{enumerate}
    
\end{frame}


\begin{frame}[fragile]{Example: External Function Bypassing Logger Method}
    
    Let's add a ``history'' attribute and a ``logger'' method to keep track of all the transactions made on the account:
    
    \begin{lstlisting}[style=Matlab]
classdef BankAccount
    properties
        balance
        history
    end
    
    methods
        function obj = BankAccount(balance)
            obj.balance = balance;
            obj.history = {};
        end
        
        function obj = deposit(obj, amount)
            obj.balance = obj.balance + amount;
            obj.history{end+1} = {'deposit', amount};
        end
    \end{lstlisting}
    
\end{frame}
        

\begin{frame}[fragile]{Example: External Function Bypassing Logger Method}
    
    \begin{lstlisting}[style=Matlab]
        function obj = withdraw(obj, amount)
            if amount > obj.balance
                error('Overdrawing not allowed!');
            end
            obj.balance = obj.balance - amount;
            obj.history{end+1} = {'withdraw', amount};
        end
        
        function logger(obj)
            for i = 1:length(obj.history)
                fprintf('%s %d\n', ...
                    obj.history{i}{1}, ... 
                    obj.history{i}{2});
            end
        end
    end
end
    \end{lstlisting}
    
\end{frame}


\begin{frame}[fragile]{Example: External Function Bypassing Logger Method}
    
    Now suppose an \textit{external} function edits the balance attribute of the BankAccount object directly, bypassing the deposit and withdraw methods and the logger method:
    
    \begin{lstlisting}[style=Matlab]
>> account = BankAccount(0);
>> account.balance = 1e9
    \end{lstlisting}
    
    This bypasses the logger method and leads to a corrupted state where the transaction history no longer reflects the actual transactions made on the account.
    
\end{frame}


\section{Private vs.\ Public}

\begin{frame}[fragile]{Private vs. Public in OOP}

    Public methods and attributes are accessible from outside the class
    
    Private methods and attributes are only accessible within the class
    
    \begin{lstlisting}[style=Matlab]
classdef MyClass
    properties
        public_attr = 0;
    end
    properties (Access = private)
        private_attr = 1;
    end
    methods
        function obj = MyClass(), end
        function public_method(obj), end
    end
    methods (Access = private)
        function private_method(obj), end
    end
end
    \end{lstlisting}
    
\end{frame}


\begin{frame}[fragile]{Private vs Public Methods and Attributes}
  \begin{itemize}
    \item MATLAB lets you define methods or attributes as private
    \item Private methods and attributes are available for internal use only, and cannot be accessed from outside the class
    \item Public methods and attributes can be freely accessed from outside the class
  \end{itemize}
\end{frame}

\begin{frame}[fragile]{Private Attributes in the BankAccount Class}
\begin{lstlisting}[style=Matlab]
classdef BankAccount
   properties (Access = private)
      balance
   end
   methods
      function obj = BankAccount(initial_balance)
         if initial_balance < 0
            error('Initial balance cannot be negative.'), end
         obj.balance = initial_balance;
      end
      function balance = get_balance(obj)
         balance = obj.balance;
end, end, end
\end{lstlisting}

  \begin{itemize}
    \item The `balance` attribute in `BankAccount` is now private
    \item It is accessed using the `get\_balance()` method, which is public
    \item External code can't modify the `balance` attribute directly
    \item This ensures that the balance is always valid and prevents external code from corrupting the object's state
  \end{itemize}
\end{frame}

\begin{frame}[fragile]{Public Methods in the BankAccount Class}
  \begin{lstlisting}[style=matlab]
classdef BankAccount
    properties(Access = private)
        balance
    end
    
    methods
        function obj = BankAccount(initial_balance)
            if initial_balance < 0
                error('Initial balance cannot be <0')
            end
            obj.balance = initial_balance;
        end
        
        function balance = get_balance(obj)
            balance = obj.balance;
        end
    
  \end{lstlisting}
\end{frame}    
\begin{frame}[fragile]{Public Methods in the BankAccount Class}
  \begin{lstlisting}[style=matlab]
        function deposit(obj, amount)
            if amount < 0
                error('Deposit amount cannot be <0')
            end
            obj.balance = obj.balance + amount;
        end
        
        function withdraw(obj, amount)
            if amount < 0
                error('Withdrawal amount cannot be <0')
            end
            if obj.balance < amount
                error('Insufficient funds')
            end
            obj.balance = obj.balance - amount;
        end
    end
end

  \end{lstlisting}
\end{frame}

\begin{frame}[fragile]{Public Methods in the BankAccount Class}
  \begin{itemize}
    \item The `deposit()` and `withdraw()` methods in `BankAccount` are public
    \item They allow external code to modify the object's state in a controlled manner
    \item They perform validation to ensure that the object remains in a valid state
    \item External code cannot access the `balance` attribute directly, so it cannot corrupt the object's state
  \end{itemize}
\end{frame}

\section{Operator overloading}

\begin{frame}[fragile]{Operator Overloading in Object-Oriented Programming}
Operator overloading is a technique in object-oriented programming that allows us to define the behavior of operators (+, -, *, /, etc.) for user-defined objects.

In MATLAB, operator overloading is achieved by defining methods (with specific names) in a class that correspond to the operator being used.

For example, the addition operator (+) can be overloaded by defining the `plus()` method in a class. 

Consider the following example:
\end{frame}


\begin{frame}[fragile]{Operator Overloading in Object-Oriented Programming}
Suppose we have a class that represents a 2D vector. We can overload the addition operator (+) to allow adding two vectors.

\begin{lstlisting}[style=matlab]
classdef Vector
    properties
        x
        y
    end
    methods
        function obj = Vector(x, y)
            obj.x = x; obj.y = y; end
        function out = plus(v1, v2)
            out = Vector(v1.x + v2.x, v1.y + v2.y); end
    end
end

\end{lstlisting}

Here, the `plus()` method takes another `Vector` object as an input, adds the corresponding components of the two vectors, and returns a new `Vector` object that represents the sum of the two vectors.
\end{frame}

\begin{frame}[fragile]{Using Overloaded Operators}
Let's create two `Vector` objects and add them together using the overloaded addition operator (+):

\begin{lstlisting}[style=matlab]
>> v1 = Vector(1, 2);
>> v2 = Vector(3, 4);
>> v3 = v1 + v2
\end{lstlisting}

Output: \begin{verbatim}v3 = 
  Vector with properties:
    x: 4
    y: 6\end{verbatim}

Here, we create two `Vector` objects, `v1` and `v2`, and add them together using the overloaded addition operator `+`. The resulting `Vector` object, `v3`, has components `(4, 6)`.

\end{frame}


\begin{frame}[fragile]{Operator Overloading in Object-Oriented Programming}
We can also overload other operators, such as the multiplication operator (.*), to perform scalar multiplication of a vector.

\begin{lstlisting}[style=matlab]
classdef Vector
    properties
        x
        y
    end
    methods
        function obj = Vector(x, y)
            obj.x = x; obj.y = y; end
        function out = plus(v1, v2)
            out = Vector(v1.x + v2.x, v1.y + v2.y); end
        function out = times(obj, k)
            out = Vector(obj.x .* k, obj.y .* k); end
end, end
\end{lstlisting}

By overloading operators, we can make our code more intuitive and expressive, and allow our user-defined objects to behave more like built-in types.
\end{frame}

\begin{frame}[fragile]{Incomplete List of Overloaded Operators in MATLAB}
\begin{center}
\begin{tabular}{ccc}
\begin{tabular}{ccc}
\rowcolor{gray!30}
{Operator} & Function & Description \\
\rowcolor{gray!0}
\texttt{+} & \texttt{plus} & Addition \\
\rowcolor{gray!10}
\texttt{-} & \texttt{minus} & Subtraction \\
\rowcolor{gray!0}
\texttt{*} & \texttt{mtimes} & Matrix multiplication \\
\rowcolor{gray!10}
\texttt{/} & \texttt{mrdivide} & Matrix right division \\
\rowcolor{gray!0}
\texttt{.*} & \texttt{times} & Multiplication \\
\rowcolor{gray!10}
\texttt{./} & \texttt{rdivide} & Right division \\
\rowcolor{gray!0}
\texttt{.\textasciicircum} & \texttt{power} & Exponentiation \\
\rowcolor{gray!10}
\texttt{==} & \texttt{eq} & Equal to \\
\rowcolor{gray!0}
\texttt{\textasciitilde=} & \texttt{ne} & Not equal to \\
\rowcolor{gray!10}
\texttt{>} & \texttt{gt} & Greater than \\
\rowcolor{gray!0}
\texttt{>=} & \texttt{ge} & Greater than or equal to \\
\rowcolor{gray!10}
\texttt{<} & \texttt{lt} & Less than \\
\rowcolor{gray!0}
\texttt{<=} & \texttt{le} & Less than or equal to \\
\end{tabular}

\end{tabular}
\end{center}
\end{frame}



\end{document}
